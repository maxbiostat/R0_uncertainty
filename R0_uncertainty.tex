\documentclass[11pt]{article}
\usepackage{calc}
\usepackage{color}
\usepackage{amsfonts}
\usepackage{latexsym}
\usepackage{placeins}
\usepackage{hyperref}
\usepackage[a4paper,top=5.5cm,bottom=4cm,left=2.5cm, right=2.5cm,foot=1cm]{geometry}
\pagenumbering{gobble}
\usepackage{setspace,relsize}               
\usepackage{moreverb}                        
\usepackage{url}
\hypersetup{colorlinks=true,citecolor=blue}
\usepackage{mathtools} 
\usepackage{amsthm}
\usepackage{amssymb}
\usepackage{indentfirst}
% \usepackage{todonotes}
\usepackage[numbers,square]{natbib}
\bibliographystyle{plain}
\usepackage[pdftex]{lscape}
\usepackage{authblk}
\usepackage{amsmath}
\usepackage[cp1250]{inputenc}
\usepackage[OT4]{fontenc}

\addtolength{\voffset}{-3.5cm} \addtolength{\textheight}{4cm}

\renewcommand{\refname}{\large{\textbf{ Bibliography}}}
\renewcommand\Authfont{\scshape\small}
\renewcommand\Affilfont{\itshape\small}
\newcommand{\keywords}[1]{\noindent{\large{\bf Keywords:}} #1\\}
\setlength{\affilsep}{1em}
\newcommand{\smalllineskip}{\baselineskip=15pt}
\newcommand{\emailaddress}[1]{{\sf#1}}
\newcommand{\speaker}[1]{\author{\underline{#1}}}
\newcommand{\speakeraffil}[1]{\affil{#1}}
\let\LaTeXtitle\title
\renewcommand{\title}[1]{\LaTeXtitle{\Large{\textbf{#1}}}}

% Title Page
\title{Some remarks on the uncertainty analysis of $R_0$ in the SIR model}

\speaker{Luiz Max F. de Carvalho} % Write speaker name here

\author[1]{Daniel A. M. Villela}
\author[2]{Flavio Coelho}
\author[1]{Leonardo S. Bastos}

%%AFFILIATIONS
\speakeraffil{Program for Scientific Computing (PROCC), Oswaldo Cruz Foundation, Brazil,\,\emailaddress{lmax.procc@gmail.com}} % Write affiliation/s of the speaker here.

\affil[2]{School of Applied Mathematics, Getulio Vargas Foundation (FGV), Brazil,\,\emailaddress{fccoelho@fgv.br}} % Write affiliation/s of the first co-author here, if there is any. If not, remove the line.

\date{\vspace{-6ex}} % Do not modify this line


\DeclareMathOperator*{\argmin}{arg\,min}
\DeclareMathOperator*{\argmax}{arg\,max}
\newtheorem{theo}{Theorem}[]
\newtheorem{proposition}{Proposition}[]
\newtheorem{remark}{Remark}[]
\setcounter{theo}{0} % assign desired value to theorem counter
\begin{document}
\maketitle

\begin{abstract}

Key-words: Basic reproductive number; uncertainty; logarithmic pooling; Gamma ratio distribution; . 
\end{abstract}

\section{Background}

$R_0$ is important, a key quantity in epidemic modelling.

Acknowledging uncertainty on parameter values is important.

logarithmic pooling is a nice and robust way of combining multiple sources of info.
\cite{poole2000} discuss the issue of propagating uncertainty through a deterministic model.


This begs the question, however, of in which order the pooling and propagation (inducing) operations should be performed.


\subsection{SIR model}

\begin{eqnarray*}
\frac{dS}{dt}&=& - \beta SI\\
\frac{dI}{dt}&=&  \beta SI - \gamma I\\
\frac{dR}{dt}&=& \gamma I 
\end{eqnarray*} 
where  $S(t) + I(t) + R(t) = N \quad \forall t$, $\beta$ is the transmission (infection) rate and $\gamma$ is the recovery rate.

\begin{equation}
\label{eq:r0def}
R_0 = \frac{\beta N}{\gamma}. 
\end{equation}

\subsection{Uncertainty analysis}

$p(\beta, \gamma)$

$M(\cdot)$

$M(p(\beta, \gamma)) = p(R_0)$

For simplicity we will assume that $p(\beta, \gamma) = p(\beta)p(\gamma)$.

uncertainty about parameters can be represented by Gamma distributions.
\begin{eqnarray*}
f_{\beta}(b) &=& \frac{1}{\Gamma(k_1)\theta_1^{k_1}} b^{k_1} exp (- \frac{b}{\theta_1} ) \\
f_{\gamma}(g) &=& \frac{1}{\Gamma(k_2)\theta_2^{k_2}} g^{k_2} exp (- \frac{g}{\theta_2})
\end{eqnarray*}

\subsection{Logarithmic pooling} 


\section{Induce-then-pool or pool-then-induce?}


\subsection{The Gamma ratio distribution}

To derive the distribution, we begin by noting that for $N > 1$, the distribution of $\beta^{\ast} = \beta N$ is a Gamma distribution with parameters $k_1$ and $N\theta_1$.
Under the assumption of independence $p(\beta^{\ast}, \gamma) = p(\beta^{\ast})p(\gamma)$, thus
\begin{eqnarray}
% R_0 &=& \frac{\beta^{\ast}}{\gamma}\\
R_0 &=& \beta^{\ast}/\gamma\\
f_{R_0}(r) &=& A \int_{0}^{\infty} \gamma(\gamma r)^{k_1 -1} e^{-\frac{\gamma r}{N\theta_1}} \gamma^{k_2 -1} e^{-\frac{\gamma}{\theta_2}} d\gamma \\
A &=& \frac{1}{\Gamma(k_1)(N\theta_1)^{k_1}\Gamma(k_2)\theta_2^{k_2}}
\end{eqnarray}
Rearranging, yields
\begin{eqnarray}
\label{eq:toint}
f_{R_0}(r) &=& A \int_{0}^{\infty} r^{k_1 -1} \gamma^{k_1 + k_2 -1} e^{-B\gamma} d\gamma \\
        B  &=& \frac{\theta_2 r + N\theta_1}{N\theta_1\theta_2}
\end{eqnarray}

\begin{eqnarray}
\label{eq:density}
f_{R_0}(r) &=& \phi\times\space r^{k_1-1} (\theta_2 r + N\theta_1)^{-(k_1 + k_2)} \\
% f_{R_0}(r) &=& \phi\frac{r^{k_1-1}}{(\theta_2 r + N\theta_1)^{(k_1 + k_2)}}\\
\label{eq:normconst}
\phi &=&  \frac{(N\theta_1\theta_2)^{k1+k2}}{\mathcal{B}(k_1, k_2)(N\theta_1)^{k_1}\theta_2^{k_2} }
\end{eqnarray}
where $\mathcal{B}(a, b) = \Gamma(a + b)/\Gamma(a)\Gamma(b)$ is the Beta function  and $\phi$ is the normalisation constant.
The probability distribution in~(\ref{eq:density}) will be called Gamma ratio distribution henceforth.
The expectation of the Gamma ratio distribution is then
\begin{align}
\label{eq:expR0}
E(R_0) &= \int_{0}^{\infty}rf_{R0}(r)dr \\
       &= \frac{N\theta_1}{\theta_2}\frac{k_1}{(k_2-1)}
\end{align}
and its variance can be computed as
\begin{eqnarray}
\label{eq:var1}
Var(R_0) &=& E(R_0^2) - E(R_0)^2  \\
\label{eq:var2}
 &=& \left(\frac{N\theta_1}{\theta_2}\right)^2\frac{(k_1+k_2-1)k_1}{(k_2-2)(k_2-1)^2}
\end{eqnarray}
which only exists for $k_2 > 2$.

The mode is 
\begin{equation}
\label{eq:mode}
\frac{N\theta_1}{\theta_2}\frac{k_1 - 1}{(k_2 + 1)}
\end{equation}

For a slightly different derivation, based on generalised Gamma distributions, see~\cite{Coelho2007}.


Pooling Gammas gives Gamma

Pooling Gamma ratios gives Loch Ness monster

Compare the two (tails, concavity, etc).




\section{Application: $R_0$ for Ebola in West Africa}
[SCAN THE LITERATURE FOR BAYESIAN ESTIMATES]
\section*{Acknowledgements}
\bibliography{R0}

% \begin{figure}[!ht]
% \centering
% \includegraphics[width=\textwidth, height = 15cm]{figures/}
% \caption{\textbf{}.
% }
% \label{fig:}
% \end{figure}
%%
% \begin{figure}
% \hfill
% \subfigure[Title A]{\includegraphics[width=5cm]{img1}}
% \hfill
% \subfigure[Title B]{\includegraphics[width=5cm]{img2}}
% \hfill
% \caption{\textbf{}.
% }
% \label{fig:}
% \end{figure}
\end{document}          
